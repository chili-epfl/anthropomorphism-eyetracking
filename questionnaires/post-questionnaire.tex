\documentclass[a4paper,10pt,BCOR10mm,oneside,headsepline]{scrartcl}
\usepackage[english]{babel}
\usepackage[utf8]{inputenc}
\usepackage{wasysym}% provides \ocircle and \Box
\usepackage{enumitem}% easy control of topsep and leftmargin for lists
\usepackage{color}% used for background color
\usepackage{forloop}% used for \Qrating and \Qlines
\usepackage{ifthen}% used for \Qitem and \QItem
\usepackage{typearea}
\areaset{17cm}{26cm}
\setlength{\topmargin}{-1cm}
\usepackage{scrpage2}
\pagestyle{scrheadings}
\ihead{Post-questionnaire --- Participant ID:}
\ohead{\pagemark}
\chead{}
\cfoot{}

%% \Qq = Questionaire question. Oh, this is just too simple. It helps
%% making it easy to globally change the appearance of questions.
\newcommand{\Qq}[1]{\textbf{#1}}

%% \QO = Circle or box to be ticked. Used both by direct call and by
%% \Qrating and \Qlist.
\newcommand{\QO}{$\Box$}% or: $\ocircle$

%% \Qrating = Automatically create a rating scale with NUM steps, like
%% this: 0--0--0--0--0.
\newcounter{qr}
\newcommand{\Qrating}[1]{\QO\forloop{qr}{1}{\value{qr} < #1}{---\QO}}

%% \Qline = Again, this is very simple. It helps setting the line
%% thickness globally. Used both by direct call and by \Qlines.
\newcommand{\Qline}[1]{\noindent\rule{#1}{0.6pt}}

%% \Qlines = Insert NUM lines with width=\linewith. You can change the
%% \vskip value to adjust the spacing.
\newcounter{ql}
\newcommand{\Qlines}[1]{\forloop{ql}{0}{\value{ql}<#1}{\vskip0em\Qline{\linewidth}}}

%% \Qlist = This is an environment very similar to itemize but with
%% \QO in front of each list item. Useful for classical multiple
%% choice. Change leftmargin and topsep accourding to your taste.
\newenvironment{Qlist}{%
\renewcommand{\labelitemi}{\QO}
\begin{itemize}[leftmargin=1.5em,topsep=-.5em]
}{%
\end{itemize}
}

%% \Qtab = A "tabulator simulation". The first argument is the
%% distance from the left margin. The second argument is content which
%% is indented within the current row.
\newlength{\qt}
\newcommand{\Qtab}[2]{
\setlength{\qt}{\linewidth}
\addtolength{\qt}{-#1}
\hfill\parbox[t]{\qt}{\raggedright #2}
}

%% \Qitem = Item with automatic numbering. The first optional argument
%% can be used to create sub-items like 2a, 2b, 2c, ... The item
%% number is increased if the first argument is omitted or equals 'a'.
%% You will have to adjust this if you prefer a different numbering
%% scheme. Adjust topsep and leftmargin as needed.
\newcounter{itemnummer}
\newcommand{\Qitem}[2][]{% #1 optional, #2 notwendig
\ifthenelse{\equal{#1}{}}{\stepcounter{itemnummer}}{}
\ifthenelse{\equal{#1}{a}}{\stepcounter{itemnummer}}{}
\begin{enumerate}[topsep=2pt,leftmargin=2.8em]
\item[\textbf{\arabic{itemnummer}#1.}] #2
\end{enumerate}
}

%% \QItem = Like \Qitem but with alternating background color. This
%% might be error prone as I hard-coded some lengths (-5.25pt and
%% -3pt)! I do not yet understand why I need them.
\definecolor{bgodd}{rgb}{0.8,0.8,0.8}
\definecolor{bgeven}{rgb}{0.9,0.9,0.9}
\newcounter{itemoddeven}
\newlength{\gb}
\newcommand{\QItem}[2][]{% #1 optional, #2 notwendig
\setlength{\gb}{\linewidth}
\addtolength{\gb}{-5.25pt}
\ifthenelse{\equal{\value{itemoddeven}}{0}}{%
\noindent\colorbox{bgeven}{\hskip-3pt\begin{minipage}{\gb}\Qitem[#1]{#2}\end{minipage}}%
\stepcounter{itemoddeven}%
}{%
\noindent\colorbox{bgodd}{\hskip-3pt\begin{minipage}{\gb}\Qitem[#1]{#2}\end{minipage}}%
\setcounter{itemoddeven}{0}%
}
}

%%%%%%%%%%%%%%%%%%%%%%%%%%%%%%%%%%%%%%%%%%%%%%%%%%%%%%%%%%%%
%% End of questionnaire command definitions %%
%%%%%%%%%%%%%%%%%%%%%%%%%%%%%%%%%%%%%%%%%%%%%%%%%%%%%%%%%%%%

\begin{document}

%\begin{center}
%\textbf{\huge Anthropomorphism \& Eye-tracking}
%\end{center}\vskip1em

\noindent You have 10 minutes maximum to fill this questionnaire. If the
meaning of a question is unclear to you, please leave it out.

\noindent \textit{This questionnaire is anonymous.}

\vskip 2em


\Qitem{ \Qq{In your opinion, the three tasks that you watched were more:} 
    \begin{Qlist}
    \item a robot kind of task
    \item a human kind of task
    \end{Qlist}
}

\minisec{Based on the videos you have watched, please rate your impression of robots on these scales:}
\vskip 1em

\Qitem{ \Qq{Fake} \Qrating{5} \Qq{Natural} \hfill {\QO{} No opinion}}
\Qitem{ \Qq{Machinelike} \Qrating{5} \Qq{Humanlike} \hfill {\QO{} No opinion}}
\Qitem{ \Qq{Unconscious} \Qrating{5} \Qq{Conscious} \hfill {\QO{} No opinion}}
\Qitem{ \Qq{Artificial} \Qrating{5} \Qq{Lifelike} \hfill {\QO{} No opinion}}
\Qitem{ \Qq{Moving rigidly} \Qrating{5} \Qq{Moving elegantly} \hfill {\QO{} No opinion}}

\Qitem{ \Qq{Dislike} \Qrating{5} \Qq{Like} \hfill {\QO{} No opinion}}
\Qitem{ \Qq{Unfriendly} \Qrating{5} \Qq{Friendly} \hfill {\QO{} No opinion}}
\Qitem{ \Qq{Unkind} \Qrating{5} \Qq{Kind} \hfill {\QO{} No opinion}}
\Qitem{ \Qq{Unpleasant} \Qrating{5} \Qq{Pleasant} \hfill {\QO{} No opinion}}
\Qitem{ \Qq{Awful} \Qrating{5} \Qq{Nice} \hfill {\QO{} No opinion}}

\Qitem{ \Qq{Incompetent} \Qrating{5} \Qq{Competent} \hfill {\QO{} No opinion}}
\Qitem{ \Qq{Ignorant} \Qrating{5} \Qq{Knowledgeable} \hfill {\QO{} No opinion}}
\Qitem{ \Qq{Irresponsible} \Qrating{5} \Qq{Responsible} \hfill {\QO{} No opinion}}
\Qitem{ \Qq{Unintelligent} \Qrating{5} \Qq{Intelligent} \hfill {\QO{} No opinion}}
\Qitem{ \Qq{Foolish} \Qrating{5} \Qq{Sensible} \hfill {\QO{} No opinion}}


\minisec{Based on the videos you have watched, do you agree or disagree with the following sentences:}
\vskip 1em

\Qitem{ \Qq{Robots can be curious}    \\ \Qtab{2.5cm}{Disagree strongly \Qrating{5} Agree strongly} \QO{} No opinion} \vskip.5em
\Qitem{ \Qq{Robots can be friendly}    \\ \Qtab{2.5cm}{Disagree strongly \Qrating{5} Agree strongly} \QO{} No opinion} \vskip.5em 
\Qitem{ \Qq{Robots can be fun-loving}  \\ \Qtab{2.5cm}{Disagree strongly \Qrating{5} Agree strongly} \QO{} No opinion} \vskip.5em 
\Qitem{ \Qq{Robots can be sociable}    \\ \Qtab{2.5cm}{Disagree strongly \Qrating{5} Agree strongly} \QO{} No opinion} \vskip.5em 
\Qitem{ \Qq{Robots can be trusting}    \\ \Qtab{2.5cm}{Disagree strongly \Qrating{5} Agree strongly} \QO{} No opinion} \vskip.5em 
\Qitem{ \Qq{Robots can be aggressive}  \\ \Qtab{2.5cm}{Disagree strongly \Qrating{5} Agree strongly} \QO{} No opinion} \vskip.5em 
\Qitem{ \Qq{Robots can be distractible}\\ \Qtab{2.5cm}{Disagree strongly \Qrating{5} Agree strongly} \QO{} No opinion} \vskip.5em 
\Qitem{ \Qq{Robots can be impatient}   \\ \Qtab{2.5cm}{Disagree strongly \Qrating{5} Agree strongly} \QO{} No opinion} \vskip.5em 
\Qitem{ \Qq{Robots can be jealous}     \\ \Qtab{2.5cm}{Disagree strongly \Qrating{5} Agree strongly} \QO{} No opinion} \vskip.5em 
\Qitem{ \Qq{Robots can be nervous}     \\ \Qtab{2.5cm}{Disagree strongly \Qrating{5} Agree strongly} \QO{} No opinion} \vskip.5em 

\Qitem{ \Qq{Robots can be broadminded} \\ \Qtab{2.5cm}{Disagree strongly \Qrating{5} Agree strongly} \QO{} No opinion} \vskip.5em
\Qitem{ \Qq{Robots can be humble}      \\ \Qtab{2.5cm}{Disagree strongly \Qrating{5} Agree strongly} \QO{} No opinion} \vskip.5em 
\Qitem{ \Qq{Robots can be organized}   \\ \Qtab{2.5cm}{Disagree strongly \Qrating{5} Agree strongly} \QO{} No opinion} \vskip.5em 
\Qitem{ \Qq{Robots can be polite}      \\ \Qtab{2.5cm}{Disagree strongly \Qrating{5} Agree strongly} \QO{} No opinion} \vskip.5em 
\Qitem{ \Qq{Robots can be thorough}    \\ \Qtab{2.5cm}{Disagree strongly \Qrating{5} Agree strongly} \QO{} No opinion} \vskip.5em 
\Qitem{ \Qq{Robots can be cold}        \\ \Qtab{2.5cm}{Disagree strongly \Qrating{5} Agree strongly} \QO{} No opinion} \vskip.5em 
\Qitem{ \Qq{Robots can be conservative} \\ \Qtab{2.5cm}{Disagree strongly \Qrating{5} Agree strongly} \QO{} No opinion} \vskip.5em
\Qitem{ \Qq{Robots can be hard-hearted} \\ \Qtab{2.5cm}{Disagree strongly \Qrating{5} Agree strongly} \QO{} No opinion} \vskip.5em
\Qitem{ \Qq{Robots can be rude}        \\ \Qtab{2.5cm}{Disagree strongly \Qrating{5} Agree strongly} \QO{} No opinion} \vskip.5em
\Qitem{ \Qq{Robots can be shallow}     \\ \Qtab{2.5cm}{Disagree strongly \Qrating{5} Agree strongly} \QO{} No opinion} \vskip.5em

\vskip 3em
\noindent Thank you for your participation!

\end{document}
